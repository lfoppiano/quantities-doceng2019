%%%% Proceedings format for most of ACM conferences (with the exceptions listed below) and all ICPS volumes.
\documentclass[sigconf]{acmart}
%%%% As of March 2017, [siggraph] is no longer used. Please use sigconf (above) for SIGGRAPH conferences.

%%%% Proceedings format for SIGPLAN conferences 
% \documentclass[sigplan, anonymous, review]{acmart}

%%%% Proceedings format for SIGCHI conferences
% \documentclass[sigchi, review]{acmart}

%%%% To use the SIGCHI extended abstract template, please visit
% https://www.overleaf.com/read/zzzfqvkmrfzn

%%
%% \BibTeX command to typeset BibTeX logo in the docs
\AtBeginDocument{%
  \providecommand\BibTeX{{%
    \normalfont B\kern-0.5em{\scshape i\kern-0.25em b}\kern-0.8em\TeX}}}

%% Rights management information.  This information is sent to you
%% when you complete the rights form.  These commands have SAMPLE
%% values in them; it is your responsibility as an author to replace
%% the commands and values with those provided to you when you
%% complete the rights form.
\setcopyright{acmcopyright}
\copyrightyear{2019}
\acmYear{2019}
\acmDOI{10.1145/1122445.1122456}

%% These commands are for a PROCEEDINGS abstract or paper.
% \acmConference[DocEng '19]{DocEng '19: ACM Symposium on Document Engineering}{September 23--26, 2019}{Berlin, DE}
% \acmBooktitle{Woodstock '18: ACM Symposium on Neural Gaze Detection, June 03--05, 2018, Woodstock, NY}
% \acmPrice{15.00}
% \acmISBN{978-1-4503-9999-9/18/06}


%%
%% Submission ID.
%% Use this when submitting an article to a sponsored event. You'll
%% receive a unique submission ID from the organizers
%% of the event, and this ID should be used as the parameter to this command.
%%\acmSubmissionID{123-A56-BU3}

%%
%% The majority of ACM publications use numbered citations and
%% references.  The command \citestyle{authoryear} switches to the
%% "author year" style.
%%
%% If you are preparing content for an event
%% sponsored by ACM SIGGRAPH, you must use the "author year" style of
%% citations and references.
%% Uncommenting
%% the next command will enable that style.
%%\citestyle{acmauthoryear}

%%
%% end of the preamble, start of the body of the document source.
\begin{document}

%%
%% The "title" command has an optional parameter,
%% allowing the author to define a "short title" to be used in page headers.
\title{Automatic Identification and Normalisation of Physical Measurements in Scientific Literature}

%%
%% The "author" command and its associated commands are used to define
%% the authors and their affiliations.
%% Of note is the shared affiliation of the first two authors, and the
%% "authornote" and "authornotemark" commands
%% used to denote shared contribution to the research.
\author{Luca Foppiano}
\email{FOPPIANO.Luca@nims.go.jp}
\orcid{0000-0002-6114-6164}
\affiliation{%
  \institution{National Institute for Materials Science (NIMS)}
  \streetaddress{1-2-1 Sengen}
  \city{Tsukuba}
  \postcode{305-0047}
  \country{Japan}
}

\author{Laurent Romary}
\email{laurent.romary@inria.fr}
\orcid{0000-0002-0756-0508}
\affiliation{%
  \institution{Inria}
  \streetaddress{2 Simone Iff}
  \city{Paris}
  \postcode{75012}
  \country{France}
}

\author{Masashi Ishii}
\email{ISHII.Masashi@nims.go.jp}
\orcid{0000-0003-0357-2832}
\affiliation{%
  \institution{National Institute for Materials Science (NIMS)}
  \streetaddress{1-2-1 Sengen}
  \city{Tsukuba}
  \postcode{305-0047}
  \country{Japan}
}

\author{Mikiko Tanifuji}
\email{TANIFUJI.Mikiko@nims.go.jp}
\orcid{000-0001-5284-6364}
\affiliation{%
  \institution{National Institute for Materials Science (NIMS)}
  \streetaddress{1-2-1 Sengen}
  \city{Tsukuba}
  \postcode{305-0047}
  \country{Japan}
}

%%
%% By default, the full list of authors will be used in the page
%% headers. Often, this list is too long, and will overlap
%% other information printed in the page headers. This command allows
%% the author to define a more concise list
%% of authors' names for this purpose.
\renewcommand{\shortauthors}{Foppiano, et al.}

%%
%% The abstract is a short summary of the work to be presented in the
%% article.
\begin{abstract}
We present Grobid-quantities, an open source application for parsing and normalising measurements from scientific and patent literature (\cite{grobid-quantities}). Tools of this kind represent the building blocks for large-scale Text and Data Mining (TDM) systems whose goal is to understand and make unstructured information accessible through standardised methods. 
Grobid-quantities is a module built up on top of Grobid (\cite{GROBID}), a machine learning framework for parsing and structuring PDF documents. Designed to process large quantities of data, it provides a robust implementation in batch mode or via a REST API. 
The machine learning engine architecture follows the cascade approach where each model is specialised in the resolution of a specific task. The models are trained using CRF (Conditional Random Field) algorithm for extracting quantities (atomic values, intervals or lists), units (like length, weight) and different value representations (such as alphanumeric, power of 10, exponential). Identified measurements are then normalised toward the International System of Units (SI) (\cite{internationalSystemOfUnits}). 
Thanks to its high recall and reliable precision, Grobid-quantities has been integrated as a measurement-extraction engine in various TDM projects, such as Marve (Measurement Context Extraction from Text) (\cite{hundman2017measurement}), for extracting semantic measurements and meaning in Earth Science. 
At the National Institute for Materials Science (NIMS), a project for application of Grobid-quantities to discover new superconducting materials is in progress: normalised materials characteristics extracted from scientific literature are a key resource for materials informatics (MI) (\cite{foppiano2019proposal}). 
\end{abstract}

%%
%% The code below is generated by the tool at http://dl.acm.org/ccs.cfm.
%% Please copy and paste the code instead of the example below.
%%
 \begin{CCSXML}
<ccs2012>
<concept>
<concept_id>10010405.10010497.10010504.10010505</concept_id>
<concept_desc>Applied computing~Document analysis</concept_desc>
<concept_significance>500</concept_significance>
</concept>
<concept>
<concept_id>10010405.10010497.10010500.10010503</concept_id>
<concept_desc>Applied computing~Document metadata</concept_desc>
<concept_significance>300</concept_significance>
</concept>
<concept>
<concept_id>10010405.10010497.10010510.10010514</concept_id>
<concept_desc>Applied computing~Format and notation</concept_desc>
<concept_significance>300</concept_significance>
</concept>
</ccs2012>
\end{CCSXML}

\ccsdesc[500]{Applied computing~Document analysis}
\ccsdesc[300]{Applied computing~Document metadata}
\ccsdesc[300]{Applied computing~Format and notation}

%%
%% Keywords. The author(s) should pick words that accurately describe
%% the work being presented. Separate the keywords with commas.
\keywords{machine learning, tdm, measurements, physical quantities}

%%
%% This command processes the author and affiliation and title
%% information and builds the first part of the formatted document.
\maketitle

\section{Introduction}
\section{Related Work}
\section{Discover Quantities}
\section{Use cases}
\section{Conclusion}

%%
%% The acknowledgments section is defined using the "acks" environment
%% (and NOT an unnumbered section). This ensures the proper
%% identification of the section in the article metadata, and the
%% consistent spelling of the heading.
\begin{acks}
Patrice Lopez
\end{acks}

%%
%% The next two lines define the bibliography style to be used, and
%% the bibliography file.
\bibliographystyle{ACM-Reference-Format}
\bibliography{references}

%%
%% If your work has an appendix, this is the place to put it.
% \appendix

\end{document}
\endinput
%%
%% End of file `sample-sigconf.tex'.
